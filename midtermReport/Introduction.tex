\section{Introduction}
Advanced Persistent Threat (APT) attacks~\cite{fireeye:anatomy,aptsymantec} have caused many well-protect firms with significant financial\\ losses~\cite{ebay,opm,target,homedepot} and become the reason for people and companies seek safer and more comprehensive information security protections. Yahoo data breach leaked over 1 billion accounts' sensitive information. This security event caused Verizon to cut the price of the deal with Yahoo by 350 million~\cite{ya:yahooleak,aptsymantec}. These APT attacks often consist of many individual attack steps across many hosts and leverage advanced tools and malware to penetrate into an enterprise~\cite{fireeye:anatomy,aptsymantec}. These complex strategies and customized tools make APT hard to be prevented. Even though these attacks can be powerful and stealthy, one typical constrains from the attacker side is that such an attack consisted by multiple steps could leave the multiple attack footprints as ``dots''.

In order to achieve the version of connecting the dots, the first challenge is to collect and store the dots. This goal can be achieved by the system kernel listening tool which can gather and output the information of system calls and provides a comprehensive way to capture system behaviors~\cite{backtracking,backtracking2}. Unlike its alternatives (file accesses, firewall and network monitoring) that provide partial  information and are application-specific, system monitoring covers all activities among system entities (process, files, sockets, and pipes) over time, providing a global view of system activities.

The second challenge for achieving the vision of connecting the dots is to connect dots even interleaved with multiple system activities. The state-of-art~\cite{taser,backtracking,backtracking2} techniques take processes as subjects and other system entities as objects. System call events happen between system entities establish edges between subjects and objects. If an anomalous is detected, a dependency analysis is applied to  connected system call events via causality. In order to finish the dependency analysis quickly, an efficient design storage and index of log data is necessary. However, APT attacks could remain stealthy for more than half a year~~\cite{trustwave}, and monitoring attack provenance on every host in an enterprise for such a period of time (\url{~}0.5--1.0 GB per host per day) is burdensome and laborious. Therefore, gathering and storing data for dependency analysis in an enterprise is an daunting task. To mitigate the problem of overwhelmingly big-data of attack derived from system monitoring data for dependency analysis, the Backtracking and Causality Preserved Reduction (CPR) are applied to shrink the size of dependency graph.

The logistics behind the Backtracking is that the events that happened after the anomalous event can not be part of multiple attack steps of APT, and therefore these events can be removed from the dependency graph. The insight of Causality Preserved Reduction is that some events have identical dependency impact space (same subject and same objects), and therefore they can be safely aggregated.

The third challenge is that although dependency analysis reveals the causality of an anomalous event as a dependency graph, they only scratch the surface of how causality analysis can be used to enhance the defense of APT attacks: it is still slow and laborious for security analysts to filter unrelated events and recover the attack sequence from the large dependency graph. In APT attacks, many steps require downloading and installing customized applications and libraries to enterprise system to succeed, such as malware injection and privilege escalation~\cite{apt,aptmalware,defendapt} and blocking applications and libraries from entering the enterprises can counter these attacks. 
%However, existing techniques that reject software with unseen signatures are ineffective in blocking malicious software installation, since they would also reject many user's application installations that adopts a faster delivery cycle for including patches and new features. 
In this project, PageRank~\cite{pagerank} is used to infer reputation of all system entities. The libraries or applications from unknown source has lower reputation compared with those from reliable channels, such as Google, Apple, Samsung and so on. After that, we can reconstruct steps based on each entity's reputation.